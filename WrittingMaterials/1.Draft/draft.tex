\documentclass[12pt,letterpaper]{article}

% Essential packages
\usepackage[utf8]{inputenc}
\usepackage[T1]{fontenc}
\usepackage[english]{babel}
\usepackage{amsmath,amsfonts,amssymb}
\usepackage{graphicx}
\usepackage[table,xcdraw]{xcolor}
\usepackage{booktabs}
\usepackage{array}
\usepackage{multirow}
\usepackage{multicol}

% Page layout and formatting
\usepackage[margin=1in]{geometry}
\usepackage{setspace}
\usepackage{fancyhdr}
\usepackage{titlesec}
\usepackage{caption}
\usepackage{subcaption}

% References and citations
\usepackage[style=numeric,backend=bibtex,sorting=none]{biblatex}
\addbibresource{references.bib} % Create this file for your references

% Table of contents formatting
\usepackage{tocloft}
\usepackage{hyperref}
\hypersetup{
    colorlinks=true,
    linkcolor=black,
    filecolor=magenta,      
    urlcolor=blue,
    citecolor=blue
}

% Figure and table numbering
\usepackage{chngcntr}
\counterwithin{figure}{section}
\counterwithin{table}{section}

% Custom formatting for captions
\captionsetup{
    font=small,
    labelfont=bf,
    format=plain,
    justification=centering,
    margin=0.5cm
}

% Header and footer
\pagestyle{fancy}
\fancyhf{}
\rhead{\thepage}
\lhead{DRAFT--Transient Blood Rheology--DRAFT}
\renewcommand{\headrulewidth}{0.4pt}

% Title formatting
\titleformat{\section}{\large\bfseries}{\thesection}{1em}{}
\titleformat{\subsection}{\normalsize\bfseries}{\thesubsection}{1em}{}
\titleformat{\subsubsection}{\normalsize\itshape}{\thesubsubsection}{1em}{}

% Line spacing
\onehalfspacing

\begin{document}

% Title page
\begin{titlepage}
    \centering
    \vspace*{2cm}
    
    {\LARGE\bfseries DRAFT -- Main Title Here -- DRAFT\par}
    
    \vspace{1.5cm}
    
    {\large A Machine Learning Approach to Correlating Physiological Parameters with Rheological Features\par}
    
    \vspace{2cm}
    
    {\Large A Manning Smith\par}
    \vspace{0.5cm}
    {\large University of Delaware\par}
    {\large Bioinformatics and Computation Science\par}
    
    \vspace{2cm}
    
    {\large \today\par}
    
    \vfill
    
    \begin{abstract}
    \noindent
    Abstract goes here
    \end{abstract}
    
\end{titlepage}

% Table of contents
\newpage
\tableofcontents

% Introduction
\newpage
\section{Introduction}
Intro section here.

% Methods
\newpage
\section{Methods}
Methods section here.

% Results and Discussion
\newpage
\section{Results and Discussion}
Results and discussion section here.

% Future Directions
\newpage
\section{Future Directions}
Future directions section here.

% Conslusions
\newpage
\section{Conclusion}
Conclusion section here.

% Acknowledgments
\newpage
\section{Acknowledgments}

- Sean Farrington
- Norman J. Wagner

% Resources
\newpage
\section{Resources}
Resources section here.

% Accronyms
\newpage
\section{Acronyms}
\begin{itemize}
    \item GPR: Gaussian Process Regression
    \item PC: Principal Component
    \item ML: Machine Learning
    \item PCA: Principal Component Analysis
    \item CV: Cross-Validation
    \item KFCV: K-Fold Cross-Validation
    \item MSE: Mean Squared Error
    \item RMSE: Root Mean Squared Error
    \item MAE: Mean Absolute Error
    \item $R^2$: Coefficient of Determination
    \item CI: Confidence Interval
\end{itemize}

% References
\newpage
\printbibliography[title=References]

% Appendices (if needed)
\newpage
\appendix

\section{Supplementary Data}
\label{app:supplementary}

[Include supplementary figures, tables, or data here]

\end{document}