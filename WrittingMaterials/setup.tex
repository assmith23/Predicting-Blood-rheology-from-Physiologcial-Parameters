\documentclass[12pt,letterpaper]{article}

% Essential packages
\usepackage[utf8]{inputenc}
\usepackage[T1]{fontenc}
\usepackage[english]{babel}
\usepackage{amsmath,amsfonts,amssymb}
\usepackage{graphicx}
\usepackage[table,xcdraw]{xcolor}
\usepackage{booktabs}
\usepackage{array}
\usepackage{multirow}
\usepackage{multicol}

% Page layout and formatting
\usepackage[margin=1in]{geometry}
\usepackage{setspace}
\usepackage{fancyhdr}
\usepackage{titlesec}
\usepackage{caption}
\usepackage{subcaption}

% References and citations
\usepackage[style=numeric,backend=bibtex,sorting=none]{biblatex}
\addbibresource{references.bib} % Create this file for your references

% Table of contents formatting
\usepackage{tocloft}
\usepackage{hyperref}
\hypersetup{
    colorlinks=true,
    linkcolor=black,
    filecolor=magenta,      
    urlcolor=blue,
    citecolor=blue
}

% Figure and table numbering
\usepackage{chngcntr}
\counterwithin{figure}{section}
\counterwithin{table}{section}

% Custom formatting for captions
\captionsetup{
    font=small,
    labelfont=bf,
    format=plain,
    justification=centering,
    margin=0.5cm
}

% Header and footer
\pagestyle{fancy}
\fancyhf{}
\rhead{\thepage}
\lhead{Parameterizing Transient Blood Rheology}
\renewcommand{\headrulewidth}{0.4pt}

% Title formatting
\titleformat{\section}{\large\bfseries}{\thesection}{1em}{}
\titleformat{\subsection}{\normalsize\bfseries}{\thesubsection}{1em}{}
\titleformat{\subsubsection}{\normalsize\itshape}{\thesubsubsection}{1em}{}

% Line spacing
\onehalfspacing

\begin{document}

% Title page
\begin{titlepage}
    \centering
    \vspace*{2cm}
    
    {\LARGE\bfseries Parameterizing Transient Blood Rheology with Standard Blood Panel Physiology\par}
    
    \vspace{1.5cm}
    
    {\large A Machine Learning Approach to Correlating Physiological Parameters with Rheological Constitutive Models\par}
    
    \vspace{2cm}
    
    {\Large Author Name\par}
    \vspace{0.5cm}
    {\large Department/Institution\par}
    {\large University Name\par}
    
    \vspace{2cm}
    
    {\large \today\par}
    
    \vfill
    
    \begin{abstract}
    \noindent This project aims to correlate routine physiological blood test data with a time-dependent (transient) constitutive model of blood rheology. Blood rheology models describe flow and deformation under different forces and timescales, essential for simulating circulation, cardiovascular disease diagnosis, and medical device design. The goal is to express human blood's transient constitutive model parameters as functions of measurable blood panel values using machine learning approaches. The tensorial-Enhanced Structural Stress Thixotropic Viscoelastic (t-ESSTV) model parameters will be predicted from standard physiological metrics including hematocrit, cholesterol, plasma protein concentration, and other routine blood panel measurements.
    \end{abstract}
    
\end{titlepage}

% Table of contents
\newpage
\tableofcontents
\newpage

% List of figures (optional - uncomment if needed)
% \listoffigures
% \newpage

% List of tables (optional - uncomment if needed)  
% \listoftables
% \newpage

\section{Introduction}

\section{Conclusions}

This study successfully demonstrated the feasibility of predicting blood rheological parameters from standard physiological measurements using machine learning approaches. The developed models provide a bridge between detailed rheological characterization and clinical practicality, potentially enabling personalized cardiovascular risk assessment based on routine blood panel data.

The identified correlations between principal components of physiological parameters and specific rheological behaviors offer new insights into the mechanistic basis of blood flow properties. While limitations in sample size require cautious interpretation, the robust validation approaches employed support the reliability of the observed relationships.

\section{Acknowledgments}

[Add acknowledgments here]

% References
\newpage
\printbibliography[title=References]

% Appendices (if needed)
\newpage
\appendix

\section{Supplementary Data}
\label{app:supplementary}

[Include supplementary figures, tables, or data here]

\end{document}